\documentclass[12pt]{article}
\usepackage[portuges]{babel}
\usepackage{a4wide}
\usepackage[utf8]{inputenc}
\usepackage{hyperlatex}
\usepackage{enumerate}
\usepackage{macros}

\htmldirectory{pratic-html}
\htmlcharset{mnemonic}
\W\newcommand{\HlxIllegalAccent}[2]{#2}
\W\renewcommand{\HlxIcons}{}
\htmldepth{2}

%\htmltitle{Exerc�cios para Introdu��o aos Computadores}
\htmladdress{\copyright 2006--20011
  Departamento de Ci�ncia de Computadores, Faculdade de Ci�ncias,
  Universidade do Porto}

%\title{Exerc�cios para Introdu��o aos Computadores}

\author{
{\small Departamento de Ci�ncia de Computadores}\\
{\small Faculdade de Ci�ncias, Universidade do Porto}}

\date{2015}


\newcommand{\lixo}[1]{}

\begin{document}
\maketitle

\htmlrule

%\xlink{P�gina da disciplina de {\em Introdu��o aos
%    Computadores}}{http://www.ncc.up.pt/~mig/rped/hrpedag_13.html}

\htmlrule

\htmlmenu{2}

\section{Utiliza��o da ``shell''}
\setcounter{subsection}{2}
\subsection{Vari�veis}

\begin{exercicio}
Utilizar vari�veis:
  \begin{enumerate}
  \item Definir a vari�vel \texttt{D} com o caminho para a direct�rio
    corrente, p.e \verb=~/aulas/ic/unix=.

\item Usando a vari�vel que criou: 
  \begin{enumerate}
  \item liste o conte�do desse direct�rio.
  \item copie o conte�do do direct�rio para o ficheiro
    \texttt{content}.
  \end{enumerate}

\end{enumerate}
\end{exercicio}

\begin{exercicio}

Alterar vari�veis de ambiente:
 
\begin{enumerate}
\item Liste todas as vari�veis de ambiente

\item Altere a vari�vel \texttt{PATH} para incluir o direct�rio \verb=~/bin=

\item Mude a direct�rio corrente para a raiz e verifique se pode executar os
seus programas.

\end{enumerate}
\end{exercicio}

\begin{exercicio}
  Alterar o ficheiro de inicializa��o da shell \texttt{.bashrc} por
  forma a que as futuras sess�es de trabalho:
  \begin{enumerate}
  \item Possam executar comandos em \verb=~/bin= e no direct�rio em que se encontre
  \item Tenham na \emph{prompt} a hora e o direct�rio corrente. [Nota: pode
    consultar a p�gina do manual (\texttt{man bash}) e o t�pico "PROMPTING".]
\end{enumerate}
  
\end{exercicio}


\subsection{ Ficheiros de Comandos}

Nos exerc�cios seguintes deve criar ficheiros de comandos que realizem
as tarefas pedidas. Dentro do seu ficheiro/comando, os argumentos
ser�o referenciados como \texttt{\$0, \$1, \$2}, \ldots.

\medskip

\begin{exercicio}\label{inscritos}
Contar o n�mero de alunos que se increveram pela primeira vez num dado
ano ({\tt \$1}). Notar que:
\begin{itemize}
\item deve utilizar o ficheiro \texttt{passwd} (da folha 2).
\item deve encadear os comandos 
\item pode utilizar o comando \verb+cut -c <colunas>+ para mostrar s� a
  parte que interessa
\end{itemize}
  
\end{exercicio}

\begin{exercicio}
  Crie um comando para encontrar ficheiros .c, a partir de um dado
direct�rio ({\tt \$1}), que contenham uma palavra dada ({\tt \$2}). 
Notar que:
\begin{itemize}
\item deve utilizar o comando \texttt{find} com a op��o \texttt{-exec} para poder executar o \texttt{grep}
\item
o \texttt{grep} deve utilizar a op��o \texttt{-H}, para mostrar o nome dos ficheiros

\item deve executar no output um \texttt{cut} para obter apenas o endere�o de cada ficheiro
\item finalmente, utilize o \texttt{sort} para eliminar as repeti��es de ficheiros.
\end{itemize}
\end{exercicio}

\end{document}
